\documentclass{article} % For LaTeX2e
\usepackage{nips14submit_e,times}
\usepackage{hyperref}
\usepackage{url}
%\documentstyle[nips14submit_09,times,art10]{article} % For LaTeX 2.09


\title{Project Report: User-Job Suitability Measurement System}


\author{
XIN LIN \\
Department of Computer Science\\
University of Texas at Austin \\
Austin, TX 78705 \\
\texttt{jimmylin@utexas.edu} \\
%\And
%Coauthor \\
%Affiliation \\
%Address \\
%\texttt{email} \\
%\AND
%Coauthor \\
%Affiliation \\
%Address \\
%\texttt{email} \\
%\And
%Coauthor \\
%Affiliation \\
%Address \\
%\texttt{email} \\
%\And
%Coauthor \\
%Affiliation \\
%Address \\
%\texttt{email} \\
%(if needed)\\
}

% The \author macro works with any number of authors. There are two commands
% used to separate the names and addresses of multiple authors: \And and \AND.
%
% Using \And between authors leaves it to \LaTeX{} to determine where to break
% the lines. Using \AND forces a linebreak at that point. So, if \LaTeX{}
% puts 3 of 4 authors names on the first line, and the last on the second
% line, try using \AND instead of \And before the third author name.

\newcommand{\fix}{\marginpar{FIX}}
\newcommand{\new}{\marginpar{NEW}}

\nipsfinalcopy % Uncomment for camera-ready version

\begin{document}


\maketitle

\begin{abstract}
    Abstract here.
\end{abstract}

\section{Introduction}

\section{Problem Formulation}

if we assume that all job seekers are extremely knowledgeable (understand
clearly and completely the profile and requirement of every job) and rational
(never apply for the unsuitable jobs), we can directly makes use of the score
obtained in the application prediction. However, such assumption receives
little support from practical analysis, in the sense that people tend to apply
for the job positions with higher salaries and correspongly much more
capability seeking.

\subsection{Suitability Measurement As Matrix Completion}
0. failure of traditional binary classifier
1. content-based filtering
2. collaborative filtering
  2.1 nearest neighbour method
  2.2 latent factor model
3. Features-incorporated Matrix Completion
  3.1 dhillon's paper: inductive matrix completion
  3.2 propose kernel-based inductive matrix completion?
4. 

\subsection{Suitability Measurement with Prerequisites}
1. simulate course recommendaiton (by adita)

\subsection{}

\subsection{}

\section{Experiments}
\subsection{Application Prediction}
2. suitability problem 


\subsubsection*{References}



\end{document}
