\documentclass{article} % For LaTeX2e
\usepackage{nips14submit_e,times}
\usepackage{cite}
\usepackage{url,graphicx}
%\documentstyle[nips14submit_09,times,art10]{article} % For LaTeX 2.09
\usepackage[pagebackref=true,breaklinks=true,letterpaper=true,colorlinks,bookmarks=false]{hyperref}

\title{Project Report: User-Job Suitability Measurement}

\author{
XIN LIN \\
Department of Computer Science\\
University of Texas at Austin \\
Austin, TX 78705 \\
\texttt{jimmylin@utexas.edu} \\
%\And
%Coauthor \\
%Affiliation \\
%Address \\
%\texttt{email} \\
%\AND
%Coauthor \\
%Affiliation \\
%Address \\
%\texttt{email} \\
%\And
%Coauthor \\
%Affiliation \\
%Address \\
%\texttt{email} \\
%\And
%Coauthor \\
%Affiliation \\
%Address \\
%\texttt{email} \\
%(if needed)\\
}

% The \author macro works with any number of authors. There are two commands
% used to separate the names and addresses of multiple authors: \And and \AND.
%
% Using \And between authors leaves it to \LaTeX{} to determine where to break
% the lines. Using \AND forces a linebreak at that point. So, if \LaTeX{}
% puts 3 of 4 authors names on the first line, and the last on the second
% line, try using \AND instead of \And before the third author name.

\newcommand{\fix}{\marginpar{FIX}}
\newcommand{\new}{\marginpar{NEW}}

\nipsfinalcopy % Uncomment for camera-ready version

\begin{document}


\maketitle

\begin{abstract}
    Abstract here.
\end{abstract}

\section{Introduction}

\section{Problem Formulation}

if we assume that all job seekers are extremely knowledgeable (understand
clearly and completely the profile and requirement of every job) and rational
(never apply for the unsuitable jobs), we can directly makes use of the score
obtained in the application prediction. However, such assumption receives
little support from practical analysis, in the sense that people tend to apply
for the job positions with higher salaries and correspongly much more
capability seeking.

\subsection{Suitability Measurement As Matrix Completion}
% explain binary suitability measurement
For startup, we first focus on binary suitability measurement problem. That
is, we only investigate the binary association, suitable (1) or not suitable
(0), for each pair of user and job. 
% Failure of traditional binary classifier
At first glance, this problem can be naively solved by traditional binary
classifiers, e.g. logistic regression and support vector machine, or one-class
learning solvers (or data description techniques).
Nevertheless, more carefull examination reveals obvious disadvantage of such
treatment. The main drawback of using traditional binary classifier is that.
% explain matrix completion formulation TODO

\subsubsection{Content-based Filtering}

\subsubsection{Collaborative Filtering}
The developing motivation of {\it Collaborative Filtering} is that the
profiles of items or users are not always available or poorly collect in
some settings. Instead, the past histories of user-item association, also called {\it
    explicit feedback}, are easy to obtain.  

Nearest neighbour method and latent factor model are two major approaches for
Collaborative Filtering.
% 2.1 nearest neighbour method TODO


% 2.2 latent factor model TODO
This model has an underlying assumption: the real exact rating matrix are
low-rank matrix.   
Stochastic Gradient Descent and Alternating Least Square are two main
approaches to numerically achieve matrix completion task of this category.

In netflix prize, the latent factor model, combined with temporal dynamics and
bias evaluation, became a single model with the greatest prediction power.

% 2.3 weakness 
However, collaborative filtering fails to provide supervision of missing
values when only rare amount of previous association histories are provided,
especailly when the systems are on its early stage.
This is commonly referred to as {\it cold startup} problem.

\subsubsection{Features-incorporated Matrix Completion}
% 3.1 dhillon's paper: inductive matrix completion
A recently emerging paper proposed an {\it Inductive Matrix Completion} method, in
which features of items are considered while completing matrices. 
According to \cite{jain2013provable}, this feature incorporation method can be
explained as a way to provide additional support for sparse matrices. 
From this perspective, we can see it as a new approach that incorporates
advantages from both collaborative filtering and content-based filtering. 
A more recent paper \cite{natarajan2014inductive} in bioinformatics 
demonstrated successful application of inductive matrix completion on
gene-disease analytics.

% 3.2 propose kernel-based inductive matrix completion?
One possible enhancement for above inductive matrix completion is to
extend our consideration from the linear association between features and
latent factors to an version that accepts non-linear association.
By this intuition, we can name this approach as 
    {\it Kernel-based Inductive Matrix Completion}.
By taking into account non-linear relations between designed features and hidden
topics (shared latent factors), the space of latent factors can be largely
expanded and then it would be more likely to automatically detect latent
factors with higher quality. 
 
%\subsection{Scored Suitability Measurement (Multi-valued Version)}
%% this can also be used as a way of 
%A more elegant way to measure suitability between users and jobs is to
%directly predict scores based knowns. 

\subsection{Suitability Measurement with Prerequisites}
1. simulate course recommendaiton (by adita)

\section{Experiments: Application Prediction}
% introduce problem and evaluation criteria
Due to limitation of acquired data, our first experiment is oriented to
problem of application prediction. Specifically, given a set of featured users
and featured jobs, the designed system should predict whether one user will
apply for one particular job.

\subsection{Dataset}
% introduce experimented data set
The dataset utilized in this experimental project comes from a Job
Recommendation Challenge posted on
\href{http://www.kaggle.com/c/job-recommendation/data}{Kaggle.com}. The
provider of this dataset is
\href{http://www.careerbuilder.com/}{CareerBuilder.com}, one of the biggest
job recommendation service providers. This particular dataset, sized of several
Gigabytes, contains a collection of featured users, a collection of
characterized jobs and a series of application records that are divided into
training and testing split.  In this section, we will at first provide the
fundamental introduction to these three basic elements: User, Job,
Application. And then explanation are illustrated about temporal separation (concept of
window) and designed distribution for user, job and application. 

% show details each record of user
As the most essential component of job recommendation system, users are
recorded by UserID, WindowID, Split, City, State, Country, ZipCode,
DegreeType, Major, GraduationDate, WorkHistoryCount,
TotalYearsExperience, CurrentlyEmployed, ManagedOthers, ManagedHowMany.
{\it UserID} refers to the nubmering index of that indicated user.  
{\it WindowsID} is about the timing period in which this particular
applicaiton about user happened. 
{\it City, State, Country, and ZipCode} are related to the living place of
that user. 
What follows are the achievements in school. 
{\it DegreeType} shows the highest degree that user has gained from school,
{\it Major} presents the field of his/her speciality and 
{\it GraduationDate} reveals when he/she gained the highest degree.
Working and management experience comes next.
{\it WorkHistoryCount} represents how many previous jobs one has had and
{\it TotalYearsExperience} represents how many years one has been involved in
occupation. 
{\it CurrentlyEmployed} and {\it ManagedOthers} are both binary values,
indicating whether one was currently on his/her job and whether he has been in
certain management position. 
{\it ManagedHowMany} implies his management power and capability, that is, the maximum number
of people he/she has managed before.

% show details each record of job
When it comes to information of every individual job, fields like JobID,
WindowID, Title, Description, Requirements, City, State, Country, Zip5,
StartDate, EndDate are provided.
{\it JobID} is the identifying number for each particular job. 
{\it WindowID} captures the same semantics as it is in a user record --
involved timing period of one job.
{\it Title} is the name of position sepcified by corresponding corporation, which
can be significantly important since it reflects relative position and power
in one company's hierarchy. 
Description and Requirements are both textual characterization
over each particular job. 
{\it Description} provides a
characteristic overview of one particular job. 
{\it Requirements} can be
viewed the basic expectation of hiring company towards the job applicants.
Similarly to the record of each individual user, every job also has location
information, such as {\it City, State, Country, ZipCode}. 
Besides, {\it StartDate} and {\it EndDate} shows the timing information of one job,
i.e. on which day it starts and ends. 

% show details each record of application
As to each piece of application history record, the dataset contains information
about the UserID, WindowID, Split, ApplicationDate, and finally JobID. 
{\it UserID} is indexing number of the user who applied for particular job,
indexed by {\it JobID}. 
{\it WindowID} implies the timing period of that application event, while
{\it ApplicationDate} presents the specific date of application event. 
And {\it Split} labelled in which division (training or testing set) that
piece of application was placed.

% explain the window ID
In outline, the data on users, job postings, and job applications that
users have made to job postings is provided. In total, the applications span
13 weeks. All the job applications are split into 7 groups, each group
representing a 13-day window. Each 13-day window is split into two parts: The
first 9 days are the training period, and the last 4 days are the test period.
The graphical representation demonstrating such splits is illustrated below.
\begin{figure}[h]
    \begin{center}
        \includegraphics[width=4.3in,height=1.4in]{./fig/datalayout.png}
        \caption{Illustrating Diagram for Temporal Separation of the Dataset}
    \end{center}
\end{figure}

Note that each user and each job posting is randomly assigned to exactly one window.

\begin{figure}[h]
    \begin{center}
        \includegraphics[width=4.3in,height=1.8in]{./fig/assignusertowindows.png}
        \caption{}
    \end{center}
\end{figure}

Each job is assigned to a window with probability proportional to the time it
was live on the site in that window. Each user is assigned to a window with
probabilty proportional to the number of applications they made to jobs in
that window. In the above image, User1 only made submissions to jobs in
Window 1, and so was assigned to Window 1 with probability 100\%. User2,
however, made submissions to jobs in both Window 1 and Window 2, and so may
have been assigned to either Window1 or Window2.

In each window, all the job applications that users in that window made to
jobs in that window during the 9-day training period. 
And with each window, users have been split into two groups, {\it Test group}
and {\it Train group}. The Test users are those who made 5 or more
applications in the 4-day test period, and the Train users are those who did
not.

For each window, the task of prediction is which jobs in that window the Test
users applied for during the window's test period. Note that users may have
applied to jobs from other windows as well, but the only thing needed to be
predicted is which jobs they applied to in their own windows.



% some key observation from dataset

\subsection{Preprocessing} % ways to preprocessing

%  our practical part
Out of computational convenience, only the first part of dataset (WindowsID =
1) are involved in our experiment.



\subsection{Traditional binary classifier}
feature-based nearest neighbour model. behavior-based nearest neighbour model
not available.

\subsection{One-class learning solver (SVM)}

\subsection{Simple matrix factorization}
Alternating least square

\subsection{Inductive Matrix Completion}

% \subsection{ (optional) kernelized inductive matrix completion}

\section{Experiments: Suitability Evaluation}
We now start to divert our focus on investigating Application Potential to
Suitability Evaluation.

%\subsubsection*{References}
\bibliography{main}{}
\bibliographystyle{plain}


\end{document}
