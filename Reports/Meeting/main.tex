\documentclass{article} % For LaTeX2e
\usepackage{nips14submit_e,times}
\usepackage{cite}
\usepackage{enumitem}
\usepackage{algorithm,algorithmicx}
\usepackage[noend]{algpseudocode}
\usepackage{url,graphicx,amsmath}
%\documentstyle[nips14submit_09,times,art10]{article} % For LaTeX 2.09
\usepackage[pagebackref=true,breaklinks=true,letterpaper=true,colorlinks,bookmarks=false]{hyperref}

\title{Project Report: Application Prediction}

\author{
%XIN LIN \\
%Department of Computer Science\\
%University of Texas at Austin \\
%Austin, TX 78705 \\
%\texttt{jimmylin@utexas.edu} \\
%\And
%Coauthor \\
%Affiliation \\
%Address \\
%\texttt{email} \\
%\AND
%Coauthor \\
%Affiliation \\
%Address \\
%\texttt{email} \\
%\And
%Coauthor \\
%Affiliation \\
%Address \\
%\texttt{email} \\
%\And
%Coauthor \\
%Affiliation \\
%Address \\
%\texttt{email} \\
%(if needed)\\
}

% The \author macro works with any number of authors. There are two commands
% used to separate the names and addresses of multiple authors: \And and \AND.
%
% Using \And between authors leaves it to \LaTeX{} to determine where to break
% the lines. Using \AND forces a linebreak at that point. So, if \LaTeX{}
% puts 3 of 4 authors names on the first line, and the last on the second
% line, try using \AND instead of \And before the third author name.

\newtheorem{remark}{Remark}
\newcommand{\fix}{\marginpar{FIX}}
\newcommand{\new}{\marginpar{NEW}}

\nipsfinalcopy % Uncomment for camera-ready version

\begin{document}


\maketitle

\section{Optimization}
\subsection{Conventional Matrix Completion}
% 2.2 latent factor model 
The latent factor model has an underlying assumption: the real exact rating matrix are
low-rank matrix. By modelling target rating variable as 
\begin{align}
    R_{ij} = U_i^T V_j 
\end{align}
Then we are able to use least square cost to guide the optimization procedure: 
\begin{equation}
    \begin{aligned}
        &\min_{U,V} 
        && \sum_{(i,j)\in \Omega} (A_{ij} - U_i^T V_j)^2
        + \frac{\lambda}{2}(||U||_F^2 + ||V||_F^2)
    \end{aligned}
\end{equation}
where $A_{i,j}$ is the observed ground-truth entry and $\Omega$ is the index set of
entries that are observed. 

\subsection{One-Class Matrix Completion}
\begin{equation}
    \begin{aligned}
        &\min_{U,V} 
        && \sum_{(i,j)\in \Omega} (1 - U_i^T V_j)^2
        + \frac{\lambda}{2}(||U||_F^2 + ||V||_F^2)
    \end{aligned}
\end{equation}
%\begin{remark}
%    Since some jobs have fairly limited number of previous history, then
%    One-Class Matrix Completion is expected to have bad performance.
%\end{remark}
\subsection{Conventional Inductive Matrix Completion}
% brief math introduction
Formulate the problem as that of recovering a low-rank matrix $W_*$ using
observed entries $R_{ij} = x_i^T W_{*} y_j$ and the user/job feature vectors $x_i$, $y_j$. By
factoring $W = U V^T$ , we see that this scheme constitutes a bi-linear
prediction $(x^T U_{*})(V_{*} y)$ for a new user/job pair $(x, y)$.
\begin{equation}
    \begin{aligned}
        &\min_{U,V} &&\sum_{(i,j)\in \Omega} (A_{ij} - x_i^TUV^Ty_j)^2 
        + \frac{\lambda}{2}(||U||_F^2 + ||V||_F^2)
    \end{aligned}
\end{equation}
% break through: dhillon's paper - inductive matrix completion
\begin{remark}
According to \cite{jain2013provable}, under standard set of assumptions,
alternating minimization provably converges at a linear rate to the global
optimum of two low-rank estimation problems: a) RIP measurements based general
low-rank matrix sensing, and b) low-rank matrix completion. A more recent
paper \cite{natarajan2014inductive} in bioinformatics demonstrated successful
application of such inductive matrix completion
framework on gene-disease analytics.
\end{remark}

\subsection{One-Class Inductive Matrix Completion}
\begin{equation}
    \begin{aligned}
        &\min_{U,V} &&\sum_{(i,j)\in\Omega} (1 - x_i^TUV^Ty_j)^2 
        + \frac{\lambda}{2}(||U||_F^2 + ||V||_F^2)
    \end{aligned}
\end{equation}

\subsection{One-Class Inductive Matrix Completion with Biases}
\begin{equation}
    \begin{aligned}
        &\min_{U,V} &&(1-\alpha) \sum_{A_{ij}=1} (1 - x_i^TUV^Ty_j)^2 +
        \alpha \sum_{A_{ij}=0} (0 - x_i^TUV^Ty_j)^2 + \frac{\lambda}{2}(||U||_F^2 + ||V||_F^2)
    \end{aligned}
\end{equation}

\subsection{One-Class Inductive Matrix Completion with kernel method}
% 3.2 propose kernel-based inductive matrix completion?
One possible enhancement for above inductive matrix completion is to
extend our consideration from the linear association between features and
latent factors to an version that accepts non-linear association.
By this intuition, we can name this approach as 
    {\it Kernel-based Inductive Matrix Completion}.
By taking into account non-linear relations between designed features and hidden
topics (shared latent factors), the space of latent factors can be largely
expanded and then it would be more likely to automatically detect latent
factors with higher quality. 
 
\subsection{One-Class Inductive Matrix Completion with pre-clustering}

\section{Algorithm}
% algorithm presentation
\newcommand{\xii}{\boldsymbol{x}_i}
\newcommand{\yj}{\boldsymbol{y}_j}
The algorithmic procedures for {\it One-class Inductive Matrix Completion} are
shown in Algorithm \ref{alg:AltMin}.  
\begin{algorithm}
    \caption{Alternating Minimization for 
        One-Class Inductive Matrix Completion with Biases}
    \label{alg:AltMin}
    \begin{algorithmic}[1]

\State \textbf{INPUT}: \\ \ \ a) sparse matrices $X$ and $Y$ denote features of users
and jobs. \\ \ \ b) matrix $A$ denotes partial observation of application
association with observed index set $\Omega$
\State \

\State Initialize $U_{(0)}$ and $V_{0}$ by uniform randomization
\State \textbf{Do}  
\State \ \ \
$V_{(k+1)}$ = argmin $(1-\alpha) \sum_{(i,j) \in \Omega} 
    (1- \xii^T U_{(k)} V_{(k)}^T \yj)^2 
    + \alpha \sum_{(i,j) \not \in \Omega} (0 - \xii^T U_{(k)} V_{(k)}^T \yj)^2$

\State \  \
$U_{(k+1)}$ = argmin $(1-\alpha) \sum_{(i,j) \in \Omega}
    (1-\xii^T U_{(k)} V_{(k+1)}^T \yj)^2 +
    \alpha \sum_{(i,j) \not \in \Omega}
    (0-\xii^T U_{(k)} V_{(k+1)}^T \yj)^2$
\State \textbf{Until} Convergence. 
\State Predict values for missing entries: for some $ (i,j) \not \in \Omega$,
\State 
\ \  \ $R_{ij} = 1,\ \text{if } \xii^T U_{*} V_{*}^T \yj > \text{ cutoff } \phi$
\State 
\ \  \ $R_{ij} = 0,\ \text{otherwise}$

\State \

\State \textbf{OUTPUT: } \\ 
\ \ a) Model Parameter $ U_{*}$ and $V_{*}$ 
\end{algorithmic}
\end{algorithm}

\begin{algorithm}
    \caption{Alternating Minimization for 
        One-Class Inductive Matrix Completion with Kernel Methods}
    \label{alg:kernel}
    \begin{algorithmic}[1]

\State \textbf{INPUT}: \\ \ \ a) sparse matrices $X$ and $Y$ denote features of users
and jobs. \\ \ \ b) matrix $A$ denotes partial observation of application
association with observed index set $\Omega$
\State \

\State Initialize $U_{(0)}$ and $V_{0}$ by uniform randomization
\State \

\State \textbf{OUTPUT: } \\ 
\ \ a) Model Parameter $ U_{*}$ and $V_{*}$ 
\end{algorithmic}
\end{algorithm}

\begin{algorithm}
    \caption{Alternating Minimization for 
        One-Class Inductive Matrix Completion with Pre-clustering}
    \label{alg:prec}
    \begin{algorithmic}[1]

\State \textbf{INPUT}: \\ \ \ a) sparse matrices $X$ and $Y$ denote features of users
and jobs. \\ \ \ b) matrix $A$ denotes partial observation of application
association with observed index set $\Omega$
\State \

\State Initialize $U_{(0)}$ and $V_{0}$ by uniform randomization
\State \

\State \textbf{OUTPUT: } \\ 
\ \ a) Model Parameter $ U_{*}$ and $V_{*}$ 
\end{algorithmic}
\end{algorithm}

\section{Practical Issues}
% introduce problem and evaluation criteria
Due to limitation of acquired data, our first experiment is oriented to
problem of application prediction. Specifically, given a set of featured users
and featured jobs, the designed system should predict whether one user will
apply for one particular job.

\subsection{One-class Matrix Completion with Bias} % 0.5 page
\textbf{Bias}. Bias parameter $\alpha \in (0, 1)$ is the weight to balance
optimization objective between observed and missing labels. Note that with
unbiased version of inductive one-class matrix completion, this parameter
should be set to $1$, that is $\alpha = 1$.

\textbf{Initialization}. Randomizing all components of $U_{(0)}$ and
$V_{(0)}$ to uniform distribution over $(0, 1)$ works well in our
experimentation. 

\textbf{Optimization}.
During each stage of alternating minimization, standarded conjugate gradient method is
employed to solve least square problem. 

\section{Experiments}
\subsection{Dataset and Preprocessed Data} 

\subsection{Models for Comparison}
\renewcommand{\labelenumi}{(\Roman{enumi})}
\begin{enumerate}
    \item One-Class Conventional Matrix Completion
    \item One-Class Inductive Matrix Completion
    \item One-Class Inductive Matrix Completion with Biases
    \item (TBD) One-Class Inductive Matrix Completion with Kernel Methods 
    \item (TBD) One-Class Inductive Matrix Completion with Pre-Clustering
\end{enumerate}

\subsection{Results}
% Precision and Recall performance


\begin{figure}[h]

    \caption{Precision v.s. Recall Comparison between (I) (II) (III) with various number of topics
        (a) $K=K_1$, (b) $K=K_2$, (c) $K=K_3$, (d) $K=K_4$. }
\end{figure}

\begin{figure}[h]

    \caption{Precision v.s. Recall Comparison between models with
        $\lambda=\lambda_0, \lambda_1,\lambda_2, \lambda_3$ }
\end{figure}

\begin{figure}[h]

    \caption{Precision v.s. Recall Comparison between various models with
        different preprocessed data}
\end{figure}

\section{Conclusions}
% one concluding paragraph


%\subsubsection*{References}
\bibliography{main}{}
\bibliographystyle{plain}


\end{document}
